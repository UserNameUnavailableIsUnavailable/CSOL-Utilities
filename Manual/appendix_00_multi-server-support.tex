\section{多区服挂机设置}

\section{Steam 国际服}

首先,打开 \lstinline{Controller.ps1},在脚本开头找到如下定义 \verb|$Options| 的代码段:

\begin{minted}{powershell}
$Options = @{
	"--game-root-dir" = ''
	"--launch-game-cmd" = ''
	"--game-window-title" = ''
	# 其他选项……
}
\end{minted}

\verb|--game-root-dir| 是游戏根目录,也即 \verb|Bin| 和 \verb|Data| 所在目录,在 Steam 中,点击“管理 - 浏览本地文件”即可打开游戏根目录,在地址栏中复制路径后,粘贴到该选项中即可。例如,下图中的路径为:

\begin{minted}{text}
'C:\Users\Silver\Games\Steam\steamapps\common\CSNZ'
\end{minted}

\verb|--launch-game-cmd| 是启动游戏的命令。
在游戏掉线时,会自动执行此命令以重启游戏。
启动游戏命令需要相应区服的启动器支持(如陆服的 TCGame、国际服的 Steam)。
需要指出的是,由于游戏本身需要以管理员权限运行,故为保证启动游戏权限足够,游戏启动器也需要以管理员身份运行(TCGame 默认是以管理员身份启动的,Steam 则需要手动以管理员身份运行,否则启动时会申请提权,造成游戏无法正常启动)。
Steam 支持在命令行中通过 \verb|<Steam.exe 路径> --applaunch <APPID>| 的方式启动游戏。
Counter-Strike Nexon 的 APPID 为 273110,因此,假设 Steam 安装在默认路径 \verb|C:\Program Files (x86)\Steam|,则启动命令应设置为:

\begin{minted}{text}
'"C:\Program Files (x86)\Steam\Steam.exe" --applaunch 273110'
\end{minted}

\verb|--game-window-title| 是游戏窗口标题,对于 Counter-Strike Nexon 应设置为 \verb|"Counter-Strike Nexon"|。

综上所述,对于 Steam 服,\verb|$Options| 应如下所示:

\begin{minted}{powershell}
$Options = {
	"--game-root-dir" = 'C:\Users\Silver\Games\Steam\steamapps\common\CSNZ'
	"--launch-game-cmd" = '"C:\Program Files (x86)\Steam\Steam.exe" --applaunch 273110'
	"--game-window-title" = 'Counter-Strike Nexon'
	# 其他选项……
}
\end{minted}